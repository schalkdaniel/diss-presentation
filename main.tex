\documentclass[10pt]{beamer}

\usepackage{natbib}
\bibliographystyle{apalike}

%% Packages:
%% --------------------------------------------------------------

\usetheme[progressbar=frametitle]{metropolis}
\usepackage{appendixnumberbeamer}
\usepackage{dsfont}
\usepackage{verbatim}
\usepackage{amsmath}
\usepackage{amssymb}
\usepackage{amsthm}
\usepackage{amsfonts}
\usepackage{csquotes}
\usepackage{multirow}
\usepackage{longtable}
\usepackage{enumerate}
\usepackage{bm}
\usepackage{bbm}
\usepackage{natbib}
% \usepackage[absolute,overlay]{textpos}
\usepackage{psfrag}
\usepackage{algorithm}
\usepackage{algpseudocode}
\usepackage{algpseudocodex}
\usepackage{float}
\usepackage{eqnarray}
\usepackage{arydshln}
\usepackage{tabularx}
\usepackage{placeins}
\usepackage{tikz}
\usepackage{setspace}
\usetikzlibrary{shapes,arrows,automata,positioning,calc}
\usepackage{subfig}
% \usepackage{paralist}
\usepackage{graphicx}
\usepackage{array}
\usepackage{framed}
\usepackage{excludeonly}
\usepackage{fancyvrb}
\usecolortheme{dove}
% \usefonttheme{serif}
\usepackage{xfrac}
\usepackage{xcolor}
\usepackage{mdframed}
\usepackage{caption}
\captionsetup[figure]{labelformat=empty}
\usepackage{transparent}

%\renewcommand\topstrut[1][1.2ex]{\setlength\bigstrutjot{#1}{\bigstrut[t]}}
%\renewcommand\botstrut[1][0.9ex]{\setlength\bigstrutjot{#1}{\bigstrut[b]}}


%% For speaker notes:
%% -------------------------------------------------------------

%\usepackage{pgfpages}
%\setbeameroption{show notes}
%\setbeameroption{show notes on second screen=right}

%%!! Run with `pdfpc --notes=right main.pdf

%% Custom Commands:
%% --------------------------------------------------------------

\usepackage{scalerel,stackengine}
\stackMath
\newcommand\reallywidehat[1]{%
\savestack{\tmpbox}{\stretchto{%
  \scaleto{%
    \scalerel*[\widthof{\ensuremath{#1}}]{\kern.1pt\mathchar"0362\kern.1pt}%
    {\rule{0ex}{\textheight}}%WIDTH-LIMITED CIRCUMFLEX
  }{\textheight}%
}{2.4ex}}%
\stackon[-6.9pt]{#1}{\tmpbox}%
}
\parskip 1ex

%\newcommand*{\tran}{{\mkern-1.5mu\mathsf{T}}}
%\newcommand{\AUC}{\text{AUC}}
%\newcommand{\eAUC}{\reallywidehat{\AUC}}
%\def\oplogit{\mathop{\sf logit}}
%\newcommand{\logit}[1]{\oplogit\left(#1\right)}
%\renewcommand{\ln}{\mathop{\sf ln}}
%\def\var{\mathop{\sf var}}
%\def\mean{\mathop{\sf m}}
%\def\ci{\mathop{\sf ci}}
%\def\evar{\reallywidehat{\var}}
%\newcommand{\ROC}{\text{ROC}}

% \definecolor{metropolis_theme_color}{RGB}{35,55,59}
\definecolor{metropolis_theme_color}{RGB}{42,42,42}

%% Color customizations:
\definecolor{blue}{RGB}{0,155,164}
\definecolor{lime}{RGB}{175,202,11}
\definecolor{green}{RGB}{0,137,62}
\definecolor{titleblue}{RGB}{4,58,63}
\definecolor{deepskyblue}{RGB}{0,191,255}
\definecolor{mygrey}{RGB}{240,240,240}
\definecolor{chighlight}{RGB}{139,35,35}

\setbeamercolor{frametitle}{fg=mygrey, bg=metropolis_theme_color}
\setbeamercolor{progress bar}{fg=metropolis_theme_color}
\setbeamercolor{background canvas}{bg=white}

\setbeamertemplate{frame numbering}{%
  \insertframenumber{}/\inserttotalframenumber
}
\makeatother

\setbeamertemplate{footline}[text line]{%
    \noindent\hspace*{\dimexpr-\oddsidemargin-1in\relax}%
     \colorbox{metropolis_theme_color}{
     \makebox[\dimexpr\paperwidth-2\fboxsep\relax]{
     \color{mygrey}
     \begin{minipage}{0.33\linewidth}
       \secname
     \end{minipage}\hfill
     \begin{minipage}{0.33\linewidth}
       \centering
       \insertshortauthor
     \end{minipage}\hfill
     \begin{minipage}{0.33\linewidth}
       \flushright
       \insertframenumber{}/\inserttotalframenumber
     \end{minipage}
     }}%
  \hspace*{-\paperwidth}
}

%% Shaded for nicer code highlighting:
%% ---------------------------------------------------------------

\usepackage{mdframed}
% \usepackage{verbatim}

% Define Shaded if not defined:
\makeatletter
\@ifundefined{Shaded}{%
  \newenvironment{Shaded}{\begin{snugshade}}{\end{snugshade}}%
}{}
\makeatother

\renewenvironment{Shaded}{
  \begin{mdframed}[
    backgroundcolor=mygrey,
    linecolor=metropolis_theme_color,
    rightline=false,
		leftline=false
  ]}{
  \end{mdframed}
}

%% Input custom stuff:

\renewcommand{\mathbf}{\bm}
\newcommand*{\tran}{{\mkern-1.5mu\mathsf{T}}}

\newcommand{\D}{\mathcal{D}}
\newcommand{\fh}{\hat{f}}
\newcommand{\fmh}[1][m]{\fh^{[#1]}}
\newcommand{\fmdh}{\fh^{[m-1]}}
\newcommand{\blk}{k}
\newcommand{\blK}{K}
\newcommand{\tb}{\bm{\theta}}
\newcommand{\tbh}{\hat{\bm{\theta}}}
\newcommand{\tbmh}{\hat{\bm{\theta}}^{[m]}}
\newcommand{\tbih}[1][i]{\tbh^{(#1)}}
\newcommand{\xv}{\bm{x}}
\newcommand{\pr}{r}
\newcommand{\prv}{\bm{r}}
\newcommand{\rmi}{\pr^{[m](i)}}
\newcommand{\pd}[2]{\frac{\partial #1}{\partial #2}}
\renewcommand{\xi}[1][i]{\xv^{(#1)}}
\newcommand{\yi}[1][i]{y^{(#1)}}
\newcommand{\Lxyi}{L(\yi, f(\xi))}
\newcommand{\design}{\bm{Z}}
\newcommand{\sse}{\operatorname{SSE}}
\newcommand{\argmin}{\operatorname{arg~min}}
\newcommand{\riske}{\mathcal{R}_{\text{emp}}}
\newcommand{\rmm}{\prv^{[m]}}


%% Titlepage:
%% --------------------------------------------------------------

\title{Modern approaches for component-wise boosting:}
\subtitle{Automation, efficiency, and distributed computing with application to the medical domain}
\date{March 24, 2023}
\author{\textbf{Daniel Schalk}}
\institute{\textbf{Supervisor:} Prof. Dr. Bernd Bischl\\
\textbf{Referees:} Prof. Dr. Matthias Schmid, PD Dr. Fabian Scheipl\\
\textbf{Chair of the examination panel:} Prof. Dr. Christian Heumann}
\titlegraphic{
    \vspace{3cm}\hspace{5.4cm}\transparent{0.1}\includegraphics[height=8.5cm]{figures/LMU.png}
    \vspace{-3cm}
}

%% Text:
%% --------------------------------------------------------------

\begin{document}

\maketitle

\section*{Overview}

\begin{frame}{Publications}

  List with all Publications

\end{frame}

\begin{frame}{Structure of the talk}
  \tableofcontents
\end{frame}

\section{Background}
\subsection{S1}

\begin{frame}{Terminology}
  \begin{itemize}

    \item
      $p$-dimensional covariate or feature vector $\xv = (x_1, \dots, x_p) \in \Xspace =  \Xspace_1 \times \cdots\times$ and target variable $y\in\Yspace$.

    \item
      Data set $\D = \Dset$ with $(\xi, \yi)$ sampled from an unknown probability distribution $\mathbb{P}_{xy}$.

    \item
      True underlying relationship $f : \Xspace^p \to \R$, $\xv \mapsto f(\xv)$.

    \item
      Goal of Machine Learning (ML) is to estimate a model $\fh = \argmin_{f} \riske(f | \D)$ with
      \begin{itemize}
        \item Empirical risk $\riske(f | \D) = n^{-1} \sum_{(\xv, y)\in\D} L(y, \fh(\xv))$ and
        \item Loss function $L : \Yspace\times\Yspace \to \R_+$, $(y,\yhat) \mapsto L(y,\yhat)$.
      \end{itemize}

    \item
      The inducer $\Ind : \mathbb{D} \times \hpspace \to \fspace$, $(\D, \hp) \mapsto \fh=\Ind_{\hp}(\D)$ gets a data set $\D\in\mathbb{D}$ with hyperparameters (HPs) $\hp\in\hpspace$.

  \end{itemize}
\end{frame}

\begin{frame}{Gradient boosting}
  \begin{itemize}
    \item
      Aim is to fit a model $\fh$ by conducting functional gradient descent $\fmdh = \fmh + \nu \hat{b}^{[m]}$.

    \item
      $\hat{b}$ is a base learner fitted to pseudo residuals $\rmi = -\left.\pd{\Lxyi}{f(\xi)}\right|_{f = \fmdh}$, $i \in \{1, \dots, n\}$

    \item
      The pseudo residuals contain information in which direction to move $\fmh$ for a better fit to the training data $\D$.

    \item
      The fitting is initialized with $\fh^{[0]}(\xv) = \argmin_{c\in\mathcal{Y}}\riske(c|\D)$ and repeated $M$ times or until an early stopping criterion is met.
  \end{itemize}
\end{frame}

\begin{frame}{Component-wise gradient boosting -- Basics}
  \begin{itemize}
    \item
      Compared to GB, component-wise gradient boosting (CWB) can choose from a set of $K$ base learners $b \in \{b_1, \dots, b_K\}$.

    \item
      Often, $b_1, \dots, b_K$ are chosen to be (interpretable) statistical models and hence $f$ corresponds to a generalized additive model $f(\xv) = f_0 + \sum_{k=1}^K b_k(\xv)$ with intercept $\f_0$.

    \item
      Advantages of CWB:
      \begin{itemize}
        \item
          Feasible to get fit in high-dimensional feature spaces ($p \gg n$).

        \item
          An inherent (unbiased) feature selection.

        \item
          Interpretable/explainable partial feature effects (depending on the choice of base learners).
      \end{itemize}
  \end{itemize}
\end{frame}

\begin{frame}{Component-wise gradient boosting -- Base learner}
  Closer look at base learners, design matrix, fitting etc.
\end{frame}

\begin{frame}{Component-wise gradient boosting -- Algorithm}

  \begin{algorithm}[H]
  \footnotesize
  \caption{Vanilla CWB algorithm}\label{algo:cwb}
  \vspace{0.15cm}
  \hspace*{\algorithmicindent} \textbf{Input} Train data $\D$, learning rate $\nu$, number of boosting iterations $M$, loss\\
  \hspace*{\algorithmicindent} \phantom{\textbf{Input} } function $L$, base learners $b_1, \dots, b_\blK$\\
    \hspace*{\algorithmicindent} \textbf{Output} Model $\fh = \fmh[M]$\vspace{0.1cm}
  \hrule
  \begin{algorithmic}[1]
  \Procedure{$\operatorname{CWB}$}{$\D,\nu,L,b_1, \dots, b_\blK$}
      \State Initialize: $f_0 = \fh^{[0]}(\xv) = \argmin_{c\in\mathcal{Y}}\riske(c|\D)$
      \While{$m \leq M$}
          \State $\rmi = -\left.\pd{\Lxyi}{f(\xi)}\right|_{f = \fmdh},\ \ \forall i \in \{1, \dots, n\}$%\label{algo:cwb:line:pr}
          \For{$\blk \in \{1, \dots, \blK\}$}
              \State $\tbmh_\blk = \left(\design_\blk^\tran \design_\blk + \bm{K}_\blk\right)^{-1} \design^\tran_\blk \rmm$%\label{algo:cwb:line:fitbl}
              \State $\sse_\blk = \sum_{i=1}^n(\rmi - b_\blk(\xi | \tbmh_\blk))^2$% \label{algo:cwb:line:sse}
          \EndFor
          \State $\blk^{[m]} = \argmin_{\blk\in\{1, \dots, \blK\}} \sse_\blk$% \label{algo:cwb:line:blselection}
          \State $\fmh(\xv) = \fmdh(\xv) + \nu b_{\blk^{[m]}} (\xv | \tbmh_{\blk^{[m]}})$
      \EndWhile
      \State \textbf{return} $\fh = \fh^{[M]}$
  \EndProcedure
  \end{algorithmic}
  \end{algorithm}
\end{frame}

\begin{frame}{Component-wise gradient boosting – Example}
  Example throughout this presentation is a subset of a WHO data set\footnote[frame,1]{Full description and data is available at \url{kaggle.com/datasets/kumarajarshi/life-expectancy-who}} about life expectation in years per country:

  \begin{itemize}
    \item
      Target variable is \texttt{Life.expectancy} in years.
    \item
      Features are \texttt{Country}, \texttt{Year}, \texttt{Alcohol} recorded per capital (15+) consumption (in liters of pure alcohol), and \texttt{Adult.Mortality} rates of both sexes of dying between 15 and 60 years per 1000 population.

    \item
      Numerical features \texttt{Year}, \texttt{Alcohol} and \texttt{Adult.Mortality} are modeled as P-splines \citep{eilers1996flexible} and \texttt{Country} as one-hot-encoded linear model with ridge penalty.

    %\item
      %Define setup, iterations etc.
  \end{itemize}
\end{frame}


\begin{frame}{Component-wise gradient boosting -- Example}
	\begin{figure}
		\centering
		\includegraphics[width=\textwidth]{/home/daniel/repos/diss-presentation/figures/fig-iter-0001.png}
	\end{figure}
	\addtocounter{framenumber}{0}
\end{frame}


\begin{frame}{Component-wise gradient boosting -- Example}
	\begin{figure}
		\centering
		\includegraphics[width=\textwidth]{/home/daniel/repos/diss-presentation/figures/fig-iter-0005.png}
	\end{figure}
	\addtocounter{framenumber}{-1}
\end{frame}


\begin{frame}{Component-wise gradient boosting -- Example}
	\begin{figure}
		\centering
		\includegraphics[width=\textwidth]{/home/daniel/repos/diss-presentation/figures/fig-iter-0010.png}
	\end{figure}
	\addtocounter{framenumber}{-1}
\end{frame}


\begin{frame}{Component-wise gradient boosting -- Example}
	\begin{figure}
		\centering
		\includegraphics[width=\textwidth]{/home/daniel/repos/diss-presentation/figures/fig-iter-0015.png}
	\end{figure}
	\addtocounter{framenumber}{-1}
\end{frame}


\begin{frame}{Component-wise gradient boosting -- Example}
	\begin{figure}
		\centering
		\includegraphics[width=\textwidth]{/home/daniel/repos/diss-presentation/figures/fig-iter-0020.png}
	\end{figure}
	\addtocounter{framenumber}{-1}
\end{frame}


\begin{frame}{Component-wise gradient boosting -- Example}
	\begin{figure}
		\centering
		\includegraphics[width=\textwidth]{/home/daniel/repos/diss-presentation/figures/fig-iter-0030.png}
	\end{figure}
	\addtocounter{framenumber}{-1}
\end{frame}


\begin{frame}{Component-wise gradient boosting -- Example}
	\begin{figure}
		\centering
		\includegraphics[width=\textwidth]{/home/daniel/repos/diss-presentation/figures/fig-iter-0050.png}
	\end{figure}
	\addtocounter{framenumber}{-1}
\end{frame}


\begin{frame}{Component-wise gradient boosting -- Example}
	\begin{figure}
		\centering
		\includegraphics[width=\textwidth]{/home/daniel/repos/diss-presentation/figures/fig-iter-0070.png}
	\end{figure}
	\addtocounter{framenumber}{-1}
\end{frame}


\begin{frame}{Component-wise gradient boosting -- Example}
	\begin{figure}
		\centering
		\includegraphics[width=\textwidth]{/home/daniel/repos/diss-presentation/figures/fig-iter-0090.png}
	\end{figure}
	\addtocounter{framenumber}{-1}
\end{frame}


\begin{frame}{Component-wise gradient boosting -- Example}
	\begin{figure}
		\centering
		\includegraphics[width=\textwidth]{/home/daniel/repos/diss-presentation/figures/fig-iter-0110.png}
	\end{figure}
	\addtocounter{framenumber}{-1}
\end{frame}


\begin{frame}{Component-wise gradient boosting -- Example}
	\begin{figure}
		\centering
		\includegraphics[width=\textwidth]{/home/daniel/repos/diss-presentation/figures/fig-iter-0130.png}
	\end{figure}
	\addtocounter{framenumber}{-1}
\end{frame}


\begin{frame}{Component-wise gradient boosting -- Example}
	\begin{figure}
		\centering
		\includegraphics[width=\textwidth]{/home/daniel/repos/diss-presentation/figures/fig-iter-0140.png}
	\end{figure}
	\addtocounter{framenumber}{-1}
\end{frame}


\begin{frame}{Component-wise gradient boosting -- Example}
	\begin{figure}
		\centering
		\includegraphics[width=\textwidth]{/home/daniel/repos/diss-presentation/figures/fig-iter-0145.png}
	\end{figure}
	\addtocounter{framenumber}{-1}
\end{frame}


\begin{frame}{Component-wise gradient boosting -- Example}
	\begin{figure}
		\centering
		\includegraphics[width=\textwidth]{/home/daniel/repos/diss-presentation/figures/fig-iter-0150.png}
	\end{figure}
	\addtocounter{framenumber}{-1}
\end{frame}



\begin{frame}{Part I - Efficiency}
  Name paper here and what it solves
  \begin{itemize}
    \item
      Memory consumption
  \end{itemize}
\end{frame}

\begin{frame}{Part II - Interpretable ML Framework}
  Name paper here and what it solves
  \begin{itemize}
    \item
      Easy access etc.
  \end{itemize}
\end{frame}

\begin{frame}{Part III - Distributed Component-wise Boosting}
  Name paper here and what it solves
  \begin{itemize}
    \item
      Distributed etc.
  \end{itemize}
  Misc: Distributed Model evaluation, A few slides in the end that describes the issue (sometimes not feasible to just aggregate), how we solved it for the AUC, and what can be done in the future
\end{frame}


\section{Efficiency}

\begin{frame}{Efficiency}
  About
\end{frame}

\begin{frame}{Adaption}
  About
\end{frame}

\begin{frame}{Results}
  About
\end{frame}



\section{Automation}

\begin{frame}{Automation}
  About
\end{frame}

\begin{frame}{Autocompboost}
  About
\end{frame}

\begin{frame}{Results}
  About
\end{frame}

\begin{frame}{Outlook}
  About
\end{frame}


\section{Distributed computing}

\begin{frame}{Distributed computing}
  About
\end{frame}

\begin{frame}{Adaption}
  About
\end{frame}

\begin{frame}{Results}
  About
\end{frame}

\begin{frame}{Outlook}
  About
\end{frame}


\begin{frame}[allowframebreaks]{Bla}

  Bla
  \citep[see, e.g.,][]{Pepe2003}, or \cite{delong1988comparing}

\end{frame}

\appendix

\begin{frame}[allowframebreaks]{References}

\nocite{*}
\scriptsize
\bibliography{references}

\end{frame}

\section{Backup slides}

\begin{frame}{Backup}

bla

\end{frame}

\end{document}
